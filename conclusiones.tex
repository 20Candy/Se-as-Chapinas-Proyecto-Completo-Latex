El desarrollo de la aplicación ``Señas Chapinas''  alcanzó el objetivo general de diseñar y desarrollar una herramienta tecnológica para dispositivos Android que traduce la lengua de señas guatemalteca (LENSEGUA) a texto en español. Representantes de la comunidad sorda indican que este proyecto logrará eliminar barreras de comunicación significativas de manera gradual, fomentando la inclusión social, educativa y laboral de las personas sordas en Guatemala. A lo largo del proceso, se integraron tecnologías avanzadas y un diseño centrado en el usuario para crear una solución accesible, funcional y culturalmente relevante, mejorando la calidad de vida y facilitando la interacción diaria entre personas sordas y oyentes.

\begin{enumerate}
    \item \textbf{Investigación y Comprensión de las Necesidades del Usuario:} El primer objetivo de realizar una investigación de mercado y entrevistas con usuarios finales fue cumplido de manera exitosa. A lo largo de este proceso, se recopiló información clave que permitió definir perfiles de usuario detallados y desarrollar flujos de usuario intuitivos. Las entrevistas revelaron las preferencias y necesidades específicas de la comunidad sorda en Guatemala, las cuales fueron integradas en cada fase del desarrollo del proyecto, asegurando que la solución respondiera de manera efectiva a sus requerimientos y expectativas.
    
    \item \textbf{Desarrollo de la Interfaz de Usuario:} El diseño de la interfaz de la aplicación siguió cada etapa del desarrollo estándar de UX/UI, desde la creación de prototipos hasta los ajustes finales. Durante todo el proceso, se tomó en cuenta la retroalimentación constante de los usuarios, lo que permitió ajustar y mejorar la experiencia para hacerla visualmente atractiva, accesible e intuitiva. Esto aseguró que la interfaz respondiera de manera efectiva a las necesidades y expectativas de los usuarios, cumpliendo así con el segundo objetivo específico de diseñar una interfaz centrada en la retroalimentación y los requerimientos del usuario.
    
    \item \textbf{Desarrollo en Android:}  El objetivo técnico de desarrollar la aplicación en Android se cumplió con éxito. Se utilizó una arquitectura sólida y componentes adecuados para asegurar un desarrollo eficiente y escalable. Se  integraron los servicios externos para el procesamiento de videos y la traducción de lengua de señas (LENSEGUA) a texto en español, logrando cumplir con el propósito central del proyecto: traducir la lengua de señas guatemalteca y facilitar la interacción entre personas sordas y oyentes.


    
\end{enumerate}

