El desarrollo de la aplicación ``Señas Chapinas''  alcanzó el objetivo general de desarrollar una herramienta tecnológica para dispositivos Android que traduce la lengua de señas guatemalteca (LENSEGUA) a texto en español. Representantes de la comunidad sorda indican que este proyecto logrará eliminar barreras de comunicación significativas de manera gradual, fomentando la inclusión social, educativa y laboral de las personas sordas en Guatemala.

    % MODULO DE DISEÑO =====
    \section{Diseño y Desarrollo de la aplicación} 
    Se realizaron investigaciones de mercado e entrevistas con usuarios finales para identificar las necesidades específicas de la comunidad sorda en Guatemala, lo que permitió desarrollar perfiles detallados e integrar estas necesidades en flujos de usuario intuitivos. Con un enfoque estándar de UX/UI y retroalimentación constante, se diseñó una interfaz atractiva, accesible e intuitiva, adaptada a las expectativas de los usuarios. En cuanto al desarrollo técnico, se implementó exitosamente una aplicación Android con arquitectura escalable, integrando servicios externos para traducir la lengua de señas guatemalteca (LENSEGUA) a texto en español, cumpliendo el objetivo de facilitar la interacción entre personas sordas y oyentes.



    % MODULO DE ARQUITECTURA DE RED ===
    \section{Infraestructura de red estable y segura} 
    En este proyecto se implementó un servidor robusto y eficiente que administra datos de manera segura, con configuraciones optimizadas para la ejecución de modelos de inteligencia artificial y visión por computadora en tres máquinas virtuales. Herramientas como Lynis garantizaron un nivel de seguridad elevado, alcanzando un puntaje de 6.2 en la escala CVE. Las configuraciones incluyeron apertura de puertos y scripts automatizados para el despliegue de modelos de IA y APIs, facilitando pruebas y actualizaciones. Las pruebas de carga demostraron un rendimiento sólido incluso bajo alta demanda, gracias a la integración de Gunicorn, Nginx y SQLAlchemy, asegurando estabilidad y eficiencia. Este entorno virtual controlado y seguro permite a los usuarios desarrollar y desplegar aplicaciones avanzadas con confianza.


    % MODULO DE VISION POR COMPUTADORA ====
    
    
    
    
    
    % MODULO DE CHATPGPT ====
    
    
    
    
    
    % MODULO DE LAMDA ====
        
