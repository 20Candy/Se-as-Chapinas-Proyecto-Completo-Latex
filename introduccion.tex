Este proyecto surge de la necesidad de superar las barreras de comunicación para los usuarios de LENSEGUA. ``Señas Chapinas'' es una iniciativa que consiste en el desarrollo de una aplicación móvil para Android, diseñada para traducir lengua de señas a texto utilizando tecnologías avanzadas como visión por computadora y aprendizaje profundo. Con un enfoque en el vocabulario esencial para la vida cotidiana y situaciones de emergencia, la aplicación tiene como objetivo facilitar las interacciones diarias y mejorar la calidad de vida de la comunidad sorda. 

La aplicación combina un diseño intuitivo y culturalmente relevante con una infraestructura tecnológica eficiente y segura, capaz de gestionar y procesar datos de manera efectiva. Esto asegura una experiencia de usuario accesible, funcional y preparada para crecer en el futuro, convirtiéndose en un recurso valioso para la comunicación y la integración social.

En colaboración con la comunidad sorda y con base en una retroalimentación constante, ``Señas Chapinas'' busca ser más que una aplicación; aspira a ser un recurso valioso que no solo mejore la comunicación, sino que también fomente una mayor inclusión y entendimiento dentro de la sociedad guatemalteca.
