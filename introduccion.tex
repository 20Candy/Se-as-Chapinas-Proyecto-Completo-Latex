En Guatemala, aproximadamente 300,000 personas enfrentan algún grado de discapacidad auditiva \cite{one}. En un esfuerzo por reducir la brecha lingüística existente entre la comunidad sorda y la oyente, en el 2020, el Congreso de la República de Guatemala promulgó el Decreto 3-2020. Esta legislación reconoce oficialmente a la lengua de señas de Guatemala (LENSEGUA) como un medio de comunicación compuesto por movimientos corporales y una gramática propia de las personas sordas \cite{two}. A pesar de este importante avance legislativo, su impacto hasta el momento ha sido limitado, lo que subraya la urgencia de implementar soluciones innovadoras para mejorar la comunicación y la inclusión de las personas sordas en la sociedad guatemalteca.

En este contexto, a través de este megaproyecto se busca crear un sistema de reconocimiento e interpretación de LENSEGUA. Para esto, se plantea el desarrollo de una aplicación móvil para Android que le permita a los usuarios grabar videos de personas utilizando la lengua de señas guatemalteca. Una vez grabados, se analizará automáticamente los videos para reconocer y, posteriormente, interpretar las señas al idioma español. Con un enfoque en vocabulario clave para situaciones cotidianas y de emergencia, esta herramienta busca facilitar las interacciones diarias y contribuir a mejorar la calidad de vida de la comunidad sorda en Guatemala.

La solución combina un diseño intuitivo y culturalmente relevante con tecnología avanzada que garantiza eficiencia, seguridad y procesamiento confiable de los datos. Este enfoque garantiza una experiencia de usuario accesible y funcional, preparada para responder a las necesidades cambiantes de sus usuarios. Mediante una colaboración activa y constante con la comunidad sorda, \textit{Señas Chapinas: Traductor de LENSEGUA} aspira a ser más que una herramienta tecnológica; busca convertirse en un puente hacia la inclusión y el entendimiento, promoviendo una sociedad más equitativa y accesible.