El proyecto \textit{Señas Chapinas: Traductor de LENSEGUA} emerge como una respuesta innovadora ante la necesidad crítica de desarrollar herramientas tecnológicas que faciliten la inclusión efectiva de la comunidad sorda en Guatemala. Este proyecto no solo busca abordar las barreras de comunicación existentes, sino también empoderar a las personas sordas para que puedan participar plenamente en todos los aspectos de la vida social, educativa y profesional.

La integración de tecnologías de inteligencia artificial en este contexto representa un avance significativo en la manera en que abordamos los desafíos de accesibilidad y comunicación. Al desarrollar un traductor de LENSEGUA (Lengua de Señas Guatemalteca) basado en modelos de lenguaje avanzados, este proyecto establece un precedente importante en la aplicación de soluciones tecnológicas para resolver problemáticas sociales complejas. La iniciativa no solo busca facilitar la comunicación cotidiana, sino también promover una mayor comprensión y apreciación de la riqueza lingüística y cultural de la comunidad sorda guatemalteca.