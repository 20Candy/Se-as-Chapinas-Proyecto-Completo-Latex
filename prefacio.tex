El proyecto \textit{Señas Chapinas: Traductor de Lensegua} nació de la necesidad de reducir las barreras de comunicación que enfrenta la comunidad sorda en Guatemala, quienes utilizan la lengua de señas guatemalteca, Lensegua, como su principal medio de expresión. Esta iniciativa tuvo el propósito de ofrecer una solución tecnológica que permita traducir Lensegua a texto y voz en tiempo real, ayudando a la comunidad sorda a participar en la sociedad de una manera más equitativa y autónoma.

El proyecto se divide en varios módulos clave, los cuales trabajan de manera integrada para ofrecer una plataforma robusta, eficiente y accesible, que cumpla con las expectativas de los usuarios finales y los desarrolladores. En el módulo de diseño, se priorizó crear una interfaz intuitiva y adaptable a las necesidades de la comunidad sorda, permitiendo una interacción fluida y cómoda. Por su parte, la infraestructura de red y el \textit{backend} se conciben como el núcleo técnico del sistema, asegurando estabilidad, seguridad y escalabilidad para futuras expansiones. \textbf{FALTA!}


Es nuestra esperanza que este trabajo ilumine no solo las dificultades diarias que enfrentan las personas sordas, sino también que iniciativas como \textit{Señas Chapinas} contribuyan a superar barreras comunicativas. Aspiramos a que los usuarios obtengan una visión genuina de la vida en la comunidad sorda y reconozcan la importancia de fomentar un entorno más inclusivo y accesible para todos.

