\section{Objetivo General}

Diseñar y desarrollar \textit{Señas Chapinas}, una aplicación para dispositivos Android que sea capaz de traducir la lengua de señas guatemalteca (LENSEGUA) a texto gramaticalmente correcto en español, haciendo uso de modelos avanzados de visión por computadora y procesamiento de lenguaje natural, junto con una infraestructura de red segura y eficiente.


\section{Objetivos Específicos}
\begin{itemize}

\item Desarrollar un sistema de visión por computadora destinado al reconocimiento de la lengua de señas de Guatemala. Para ello, se creará un conjunto de datos que incluya al menos veinticinco palabras de esta lengua, seleccionando aquellas que permitan formar una amplia variedad de frases de uso cotidiano.

\item Adaptar un \textit{large language model}, específicamente GPT-3.5-Turbo, para que asimile la gramática de LENSEGUA, permitiéndole interpretar oraciones que utilicen dicha gramática y las escriba correctamente en español.

\item Desarrollar una herramienta basada en LLaMA que sea capaz de comprender la gramática de LENSEGUA y generar interpretaciones coherentes en español a partir de oraciones estructuradas en dicha gramática. 

\item Configurar un servidor seguro y eficiente que optimice el uso de recursos para administrar modelos de inteligencia artificial, procesar videos, y ofrecer un ambiente accesible para pruebas y despliegues mediante APIs.

\item Realizar una investigación de mercado y entrevistas para comprender necesidades, diseñar flujos e interfaces intuitivas, y desarrollar la aplicación \textit{Señas Chapinas} integrando servicios externos para la traducción de LENSEGUA.

\end{itemize}