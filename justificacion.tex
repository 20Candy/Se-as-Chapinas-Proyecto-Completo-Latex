La comunicación es un derecho fundamental y un pilar esencial para la interacción humana, indispensable en la educación, el trabajo y la participación activa en la sociedad \cite{NacionesUnidas2024}. En Guatemala, según datos del Conadi (Consejo Nacional para la Atención de las Personas con Discapacidad) recogidos hasta marzo de 2021, existían entre 240,000 y 250,000 personas que utilizaban la lengua de señas como su principal herramienta de comunicación \cite{Conadi2021}. Sin embargo, las barreras comunicativas continúan limitando la interacción entre personas sordas y oyentes, restringiendo su acceso igualitario a oportunidades sociales y laborales. Ante esta realidad, "Señas Chapinas" surge como una solución innovadora para superar estas barreras.

Aprovechando la amplia adopción de teléfonos inteligentes en Guatemala \cite{Xie2023}, donde la mayoría de estos dispositivos operan con el sistema Android \cite{Xie2023}, este proyecto busca desarrollar una aplicación móvil que actúe como un puente de comunicación eficiente y accesible. La aplicación pretende fortalecer la autonomía de las personas sordas en diversas situaciones cotidianas y de emergencia, promoviendo al mismo tiempo el uso de LENSEGUA. Además, se sustenta en un marco legal favorable, como el Decreto del Congreso de la República de Guatemala Número 3-2020, que reconoce la Lengua de Señas de Guatemala \cite{CongresoGuatemala2022}.


Un aspecto clave de este proyecto es la integración de diversos componentes tecnológicos para que funcionen como una unidad cohesiva en lugar de elementos independientes. Esta integración resulta crucial para conectar las capacidades de los modelos de inteligencia artificial, la funcionalidad de la aplicación móvil y la infraestructura del servidor.

El propósito de "Señas Chapinas" es fomentar la inclusión laboral, facilitar el acceso a servicios esenciales y promover las interacciones sociales, contribuyendo al enriquecimiento de la comunidad guatemalteca.
Es por ello que este proyecto representa un avance significativo hacia la construcción de una sociedad que valora la diversidad y garantiza igualdad de oportunidades para todos, utilizando tecnología móvil para superar las barreras de comunicación de manera eficiente y efectiva.




