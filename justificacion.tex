La comunicación es un derecho fundamental y un elemento clave para la interacción humana, siendo indispensable en áreas como la educación, el trabajo y la participación activa en la sociedad \cite{NacionesUnidas2024}. En Guatemala, la Asociación de Sordos de Guatemala (ASEGUA) recomienda que, para garantizar una inclusión adecuada de las 250,000 personas que utilizan LENSEGUA como su principal herramienta de comunicación, deberían existir al menos 10 personas oyentes capacitadas en lengua de señas por cada una de ellas \cite{Conadi2021}. No obstante, el país está lejos de alcanzar este estándar: solo 11,500 personas han participado en cursos básicos de lengua de señas, y únicamente existen 95 intérpretes certificados en todo el territorio nacional \cite{three}.

Esta situación, según la 'Encuesta Sociolingüística de la Comunidad Sorda en Guatemala', marginaliza a la población sorda y la limita a relacionarse exclusivamente con otras personas sordas o con hijos de adultos sordos (HDAS). La escasez de personas oyentes capacitadas en LENSEGUA y de intérpretes certificados crea barreras significativas para la comunicación y la participación de personas sordas en la sociedad guatemalteca \cite{four}. Ante esta realidad, surge la idea del proyecto \textit{Señas Chapinas: Traductor de LENSEGUA}.

Aprovechando la amplia adopción de teléfonos inteligentes en Guatemala \cite{Xie2023}, donde la mayoría de estos dispositivos operan con el sistema Android \cite{Xie2023}, este proyecto busca desarrollar una aplicación móvil que actúe como un puente de comunicación eficiente y accesible. La aplicación pretende fortalecer la autonomía de las personas sordas en diversas situaciones cotidianas y de emergencia, promoviendo al mismo tiempo el uso de LENSEGUA. Además, se sustenta en un marco legal favorable, como el Decreto del Congreso de la República de Guatemala Número 3-2020, que reconoce a la Lengua de Señas de Guatemala como un medio de comunicación oficial \cite{CongresoGuatemala2022}.

Además, a través de \textit{Señas Chapinas} se busca fomentar la inclusión laboral, facilitar el acceso a servicios esenciales y promover las interacciones sociales, contribuyendo al enriquecimiento de la comunidad guatemalteca. Es por ello que este proyecto representa un avance significativo hacia la construcción de una sociedad que valora la diversidad y garantiza igualdad de oportunidades para todos, utilizando tecnología móvil para superar las barreras de comunicación de manera eficiente y efectiva.




