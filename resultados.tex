El proyecto ``Señas Chapinas'' alcanzó los objetivos planteados en cuatro áreas clave: investigación de mercado y análisis del contexto, diseño centrado en el usuario, desarrollo técnico y pruebas con usuarios. Cada uno de estos apartados proporcionó información clave para el desarrollo de la aplicación, asegurando una solución inclusiva y accesible.

\begin{enumerate}
    \item \textbf{Comprensión de la Situación de Comunidad Sorda en Guatemala}
    
    El análisis de mercado y el contexto guatemalteco revelaron que, aunque existen aplicaciones internacionales como Hand Talk Translator y SLAIT, ninguna se adapta a las necesidades específicas de la comunidad sorda en Guatemala, ni incluye LENSEGUA. Como resultado, se determinó la necesidad de desarrollar una solución local, específicamente diseñada para las barreras lingüísticas y culturales del país. Además, la revisión del Decreto 3-2020 confirmó la importancia de una herramienta que no solo facilite la comunicación entre sordos y oyentes, sino que también promueva el aprendizaje de LENSEGUA. Las entrevistas realizadas a personas sordas y a individuos en constante contacto con la comunidad sorda, junto con encuestas dirigidas a personas oyentes, subrayaron de manera contundente la necesidad de esta aplicación. Se evidenció que las barreras para las personas sordas en Guatemala son significativas y que existe un considerable desconocimiento sobre LENSEGUA, dado que el 70\% de los encuestados reconoció no estar familiarizado con esta lengua. Esta información recalca la urgencia de desarrollar una herramienta diseñada específicamente para atender estas deficiencias.
    
    \item \textbf{Diseño Centrado en el Usuario}
    
    El diseño de la aplicación fue un proceso iterativo basado en varias herramientas de diseño, como mapas de empatía, personas, diagramas de afinidad y flujos de usuario. Este proceso dio como resultado un prototipo interactivo de alto nivel desarrollado en Figma. Dicho prototipo permitió definir de manera clara la navegación y la experiencia del usuario. El prototipo integró las recomendaciones de asociaciones como En-Señas y expertos en diseño, lo que aseguró que la aplicación no solo fuera funcional, sino también accesible y fácil de usar para la comunidad sorda. 

    
    \item \textbf{Desarrollo de Aplicación Móvil para Android}
    
    El desarrollo de la aplicación fue llevado a cabo utilizando una arquitectura modular basada en el patrón MVVM, lo que permitió que la aplicación fuera fácilmente escalable y mantenible. Se implementaron componentes reutilizables, lo que aumentó la eficiencia en el desarrollo, reduciendo la redundancia en el código. Además, se integraron diversas librerías que optimizaron el rendimiento general de la aplicación. El uso de Kanban durante el desarrollo facilitó una gestión eficaz de las tareas, asegurando que se cumplieran los plazos y que las funcionalidades clave se implementaran correctamente. El desarrollo resultó en una versión funcional de la aplicación lanzada en fase de prueba cerrada en \textit{Play Store}, lo que permitió a un grupo selecto de usuarios interactuar con ella antes de su lanzamiento oficial. 


    \item \textbf{Pruebas con Usuarios Exitosas}
    
    Durante las pruebas, se presentaron los flujos críticos de la aplicación en eventos clave como la Expo UVG y una presentación especial con la asociación En-Señas. El resultado de estas pruebas fue altamente positivo, ya que los usuarios destacaron la intuitividad y facilidad de uso de la aplicación. Se registró un 90\% de éxito en las pruebas de usabilidad con En-Señas, en las cuales los usuarios pudieron completar tareas clave como la creación de cuentas, grabación de videos y traducción de señas sin dificultades importantes. Estas pruebas confirmaron que la aplicación era funcional, accesible y cumplía con las expectativas de los usuarios, demostrando cómo la investigación, el diseño y el desarrollo se integraron armoniosamente para formar un producto final exitoso. Esto reafirma que cada fase fue ejecutada correctamente, contribuyendo al logro de un resultado sólido y coherente.
    
\end{enumerate}
