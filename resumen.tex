``Señas Chapinas'' es un proyecto innovador que responde a la necesidad crítica de mejorar la comunicación para los usuarios de LENSEGUA en Guatemala. Desarrollada como una aplicación móvil para Android, esta herramienta utiliza tecnologías de visión por computadora, aprendizaje profundo y un servidor seguro y eficiente para traducir la lengua de señas guatemalteca a texto con gramática española. La aplicación está especialmente diseñada para cubrir vocabulario esencial, tanto para situaciones cotidianas como de emergencia, facilitando así las interacciones diarias y elevando la calidad de vida de la comunidad sorda.

La colaboración activa con la comunidad sorda ha sido vital en todas las etapas del proyecto, desde la concepción hasta la implementación. Esta cooperación ha permitido que ``Señas Chapinas'' se desarrolle no solo como una solución tecnológica, sino como un recurso comunitario que promueve una mayor inclusión social y entendimiento.

Con ``Señas Chapinas'', se espera establecer un precedente para futuras innovaciones en tecnologías accesibles, demostrando cómo las herramientas adecuadamente diseñadas pueden superar barreras significativas y mejorar la interacción social dentro y fuera de la comunidad sorda.