\textit{Señas Chapinas: Traductor de LENSEGUA} es un proyecto enfocado en el desarrollo de una aplicación móvil diseñada para interpretar la Lengua de Señas de Guatemala (LENSEGUA) y traducirla a español gramaticalmente correcto, facilitando la comunicación entre personas sordas y oyentes. La solución se basa en una arquitectura modular compuesta por tecnologías avanzadas que trabajan de forma integrada para garantizar robustez, eficiencia y accesibilidad.

 El módulo de visión por computadora procesa grabaciones para identificar y reconocer las señas realizadas por los usuarios. A partir de esta información, modelos de procesamiento de lenguaje natural transforman las señas detectadas en oraciones completas y coherentes, adaptándose tanto a contextos cotidianos como a situaciones de emergencia. La infraestructura de red y el \textit{backend} funcionan como el núcleo técnico que conecta todos los componentes, asegurando estabilidad, escalabilidad y seguridad. Finalmente, el diseño centrado en la experiencia del usuario proporciona una interfaz intuitiva y accesible, adaptada a las necesidades específicas de la comunidad sorda.

La colaboración activa con la comunidad sorda guatemalteca fue esencial en todas las etapas del proyecto, desde la definición de requerimientos hasta la validación final. Este enfoque participativo no solo asegura que \textit{Señas Chapinas} sea una herramienta funcional, sino también un recurso profundamente conectado con las necesidades prácticas y culturales de sus usuarios.