A lo largo del proyecto, se abordaron las necesidades específicas de la comunidad sorda en Guatemala, prestando especial atención a la adaptación cultural y a la integración de LENSEGUA, la lengua de señas guatemalteca. En esta sección, se analizarán los hallazgos clave y las lecciones aprendidas en cada fase del proyecto, destacando los desafíos y las soluciones implementadas para crear una herramienta que responda a las demandas comunicativas y educativas de sus usuarios.

Durante la fase de investigación de mercado, se analizaron aplicaciones con funcionalidades similares a las que se querían implementar en ``Señas Chapinas''. No se encontró ninguna aplicación que estuviera enfocada en las necesidades y la cultura guatemalteca ni que utilizara LENSEGUA, lo que convierte a ``Señas Chapinas'' en una propuesta pionera en este ámbito. Esta investigación también permitió identificar funcionalidades clave a partir de los comentarios de usuarios, tales como la grabación de videos con indicadores de posición, el uso de tutoriales, la posibilidad de agregar a favoritos y la calificación de traducciones. Estas funcionalidades fueron adaptadas e implementadas en la aplicación, ajustándose al contexto cultural y comunicativo de los usuarios guatemaltecos. Además, se destacaron elementos de diseño relevantes, como la simplicidad de la interfaz, una paleta de colores atractiva y botones descriptivos sin exceso de texto. Estas características fueron aplicadas cuidadosamente en la fase de diseño y se pueden observar en la versión final de la aplicación.

Paralelamente, se llevó a cabo una investigación exhaustiva sobre la situación actual de las personas sordas en Guatemala. Se descubrió que representan solo el 3\% de la población mayor de 4 años, lo que evidencia su condición de minoría. LENSEGUA fue reconocida oficialmente hace poco tiempo y su uso es más común en la capital. Sin embargo, actualmente se está llevando a cabo un proceso de estandarización y difusión de la lengua de señas en todo el país, lo que subraya la necesidad de herramientas como ``Señas Chapinas'' para apoyar este esfuerzo. La aplicación facilita la expansión del conocimiento y el uso de LENSEGUA a nivel nacional, a través del desarrollo de un diccionario de señas y retos diarios, estrategias clave para promover su difusión en todas las regiones.

Las encuestas realizadas a personas oyentes revelaron que el 70\% desconocía la existencia de LENSEGUA, y solo consideraban su aprendizaje relevante en caso de tener contacto directo con personas sordas. Esto pone en evidencia la necesidad de ``Señas Chapinas'' no solo como una herramienta de comunicación, sino también como un vehículo educativo para incrementar la visibilidad de LENSEGUA en Guatemala.

Asimismo, se descubrió una alta tasa de analfabetismo y desempleo en la comunidad sorda, reflejando las limitadas oportunidades a las que se enfrentan. Entrevistas a personas sordas y a quienes interactúan frecuentemente con ellas confirmaron las dificultades diarias en áreas como la salud, la educación y la representación legal. Las funcionalidades de ``Señas Chapinas'' están diseñadas específicamente para abordar estas barreras. Por ejemplo, la integración de un módulo de traducción que acepta vocabulario para situaciones médicas, como emergencias, se desarrolló en respuesta a necesidades identificadas en estas entrevistas, demostrando que la aplicación no solo facilita la comunicación sino que también ofrece soluciones concretas a problemas cotidianos.

Con una comprensión clara de las necesidades y del contexto, se inició el diseño de la aplicación. Se definieron los objetivos y funcionalidades clave a través de la creación de diagramas de afinidad, mapas de empatía y la identificación de escenarios de uso. Esto permitió desarrollar un prototipo que integraba las necesidades específicas identificadas en la fase de investigación. La colaboración con expertos y la retroalimentación constante permitieron iterar y mejorar los prototipos hasta obtener un diseño final intuitivo y sencillo.

Las pruebas de usabilidad fueron exitosas, con más del 75\% de éxito en la realización de cada uno de los flujos de la aplicación por parte de los usuarios. Durante la EXPO UVG y en reuniones con directivos de En-Señas, se confirmó que el diseño del logo, la elección de colores y tipografía lograban representar la identidad guatemalteca así como a la comunidad sorda. Este enfoque en la usabilidad y el diseño confirma la sinergia entre la estética y la funcionalidad de la aplicación, lo que permitió un producto final que satisface tanto las expectativas de los usuarios como los estándares de diseño.

El proceso no estuvo exento de desafíos, especialmente en lo que respecta a las expectativas y la coherencia funcional. Alcanzar un prototipo que cumpliera con las expectativas de todos los usuarios finales sin comprometer las funcionalidades clave fue complejo debido a la diversidad de opiniones de los colaboradores. Se buscó un equilibrio cuidadoso entre las sugerencias para mantener la dirección del proyecto clara. Asimismo, hubo dificultades con el contenido visual de la aplicación. Originalmente, se utilizaron imágenes de un libro de señas proporcionado por colaboradores, pero estas tuvieron que ser reemplazadas por ilustraciones hechas a mano debido a cuestiones de derechos de autor. A pesar de estos retos, las adaptaciones logradas fortalecieron la identidad visual del proyecto.

Con un diseño validado, la fase de desarrollo móvil avanzó de manera fluida, siguiendo un enfoque estructurado y sistemático. Se adoptaron estándares arquitectónicos para Android, específicamente utilizando la arquitectura MVVM (Model-View-ViewModel), que facilitó la separación de lógica de negocio y la interfaz de usuario, garantizando una mayor mantenibilidad y escalabilidad del proyecto. Se hizo un uso extensivo de componentes reutilizables definidos durante la etapa de prototipado, lo que permitió mantener la consistencia visual y funcional a lo largo de la aplicación.

La experiencia de usuario se optimizó mediante la implementación de flujos alternativos, como el uso de \textit{deeplinks} para simplificar tareas complejas, como la recuperación y cambio de contraseñas. Además, se integraron herramientas de seguridad, incluyendo el cifrado de datos sensibles y un manejo seguro de la información del usuario, siguiendo las mejores prácticas recomendadas para aplicaciones móviles. Esto no solo mejoró la experiencia del usuario, sino que también aumentó la confianza en la aplicación, asegurando la protección de sus datos.

El proceso de desarrollo se gestionó mediante la metodología Kanban, lo que facilitó una organización clara y eficiente de las tareas. Las funcionalidades se dividieron en entregas incrementales, lo que permitió revisiones y ajustes constantes. Esto dio lugar a un desarrollo ágil, flexible y centrado en los objetivos, garantizando que cada fase del proyecto se completara a tiempo y con altos estándares de calidad.

Un elemento no planificado, pero que aportó un valor significativo, fue la publicación de la aplicación en tiendas de Google. A pesar de enfrentar retos administrativos y legales, como la integración de términos y condiciones y el diseño de pantallas de promoción, este proceso fue un hito importante. Además, llevó a la creación de una página web para respaldar la aplicación, lo que ayudó a consolidar la presencia digital del proyecto y a generar mayor confianza entre los usuarios.

La aprobación de Google fue un momento crucial, ya que permitió llevar a cabo una prueba cerrada con un grupo selecto de colaboradores de En-Señas. Esta fase de pruebas fue esencial para validar la funcionalidad de la aplicación en condiciones reales, permitiendo recibir retroalimentación directa y realizar ajustes finales antes del lanzamiento oficial. La aplicación no solo cumplió con las expectativas del público objetivo, sino que también mostró un alto nivel de rendimiento y estabilidad, confirmando la efectividad del enfoque de desarrollo adoptado.

``Señas Chapinas'' ha logrado desarrollar una herramienta innovadora y adaptada a la realidad guatemalteca, enfocada en la inclusión y la comunicación efectiva de la comunidad sorda. A lo largo del proyecto, se superaron desafíos significativos que permitieron fortalecer el diseño y la funcionalidad de la aplicación, creando un producto final que no solo cubre las necesidades de la comunidad, sino que también contribuye a la visibilización y el uso de LENSEGUA en toda Guatemala.

