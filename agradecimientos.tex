Queremos expresar nuestro más sincero agradecimiento a todas las personas que han contribuido a la realización de este proyecto, cada una de las cuales ha sido fundamental en su desarrollo.

Primero, expresamos nuestra gratitud a nuestros asesores, los Ingenieros Dennis Aldana, Miguel Novella, Luis Alberto Suriano y Javier Fong, profesores de la Universidad del Valle, por su invaluable guía y apoyo a lo largo de todo el proceso de investigación y redacción de este trabajo. Su experiencia y dirección experta fueron fundamentales para superar los retos académicos y prácticos de este proyecto.

Estamos profundamente agradecidos con ASEDES, especialmente con Niurka Waleska Bendfeldt Rosada y Alain de León, por proporcionarnos materiales, entrevistas y otros recursos necesarios para llevar a cabo este trabajo. Su colaboración fue indispensable para entender mejor las necesidades y desafíos de la comunidad sorda.

Nuestro reconocimiento a las alumnas practicantes de ASEDES: Evelyn Cacao, Any Max y Ruth Amézquita, quienes generosamente permitieron que las grabáramos mientras realizaban señas, contribuyendo significativamente a la autenticidad y calidad del contenido de este proyecto.

Agradecemos la colaboración de la profesora Pamela Ramírez, quien contribuyó en el diseño del logo de la aplicación. Su trabajo fue esencial, ya que el logo desempeña un papel crucial en la identidad visual y la coherencia del diseño de la aplicación.

Finalmente, un agradecimiento especial a Antonio Barrientos, Director General de En-Señas, y a Gabriela Velázquez, maestra de En-Señas, por su apertura y disposición para compartir su conocimiento y experiencia, las cuales fueron cruciales para este proyecto.

A todos ustedes, nuestro más profundo respeto y gratitud por su apoyo y contribuciones.